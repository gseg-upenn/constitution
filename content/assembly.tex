\chapter{General Assembly}
\label{art:assembly}

\begin{enumerate}[label=\Alph*.]

	\item The General Assembly shall consist of nineteen voting seats, of which twelve are reserved for two representatives from each of the following departments:
	\begin{enumerate}[label=(\roman*)]
		\item Bioengineering
		\item Chemical and Biomolecular Engineering
		\item Computer and Informational Sciences
		\item Electrical and Systems Engineering
		\item Materials Science Engineering
		\item Mechanical Engineering and Applied Mechanics
	\end{enumerate}

	\item Each of these departments must be represented by at least one member enrolled in a residential Master's program or the Accelerated Master’s Programs in Engineering (upon completion of their undergraduate degrees, during their single degree graduate program) affiliated to the department, and at least one PhD member affiliated to the department. 

	\item Additionally, there will be one representative for each of the following programs:
	\begin{enumerate}[label=(\roman*)]
		\item Master of Integrated Product Design or MSE Integrated Product Design
		\item MSE Bioengineering
		\item MSE Biotechnology
		\item MSE Chemical and Biomolecular Engineering
		\item MSE Data Science
		\item MSE Nanotechnology
		\item MSE Robotics
	\end{enumerate}

	\item The student associations for each of the above programs shall appoint or elect, according
	to their procedures, their representatives to the General Assembly. The procedures for the
	designation of representative are acceptable so long as approved by the GSEG Executive Board.

	\item One member can represent at most one department or program during their term.

	\item The voting representatives shall:
	\begin{enumerate}
		\item Serve as a liaison between GSEG and the appointing organizations.
		\item Notify GSEG with the updates from their appointing organizations and their
		respective program.
		\item Help GSEG perform its goals by participating in GSEG's advocacy, funding,
		and programming initiatives.
	\end{enumerate}

	\item The General Assembly shall consist of one non-voting representative from the following
	student groups:
	\begin{enumerate}[label=(\roman*)]
		\item Engineering Masters Advisory Board
		\item EngiQueers
		\item Penn Graduate Women in Science and Engineering
	\end{enumerate}

	\item The Executive Board may add non-voting representatives to the General Assembly 
	when it deems necessary. These may include representatives for cultural and identity
	student groups.  

	\item The term of a General Assembly representative shall be from September 1 of a year to 
	May 30 of the following year.

	\item The voting representatives, GSEG representatives to GAPSA as described in 
	Article~\ref{art:gapsa-reps}, and the members of the Executive Board as described in 
	Article~\ref{art:exec} shall be the voting members of the General Assembly. Each voting member
	of the General Assembly	shall have one vote in all matters during their terms. 

	\item The year for GSEG starts on May 1 and ends on April 30. The year shall be divided into
	three semesters:
	\begin{enumerate}[label=(\roman*)]
		\item Summer semester comprising of the months of May, June, July, and August. 
		\item Fall semester comprising of the months of September, October, November, and December.
		\item Spring semester comprising of the months of January, February, March, and April.
	\end{enumerate}
	The General Assembly shall meet at least once in each month of the Fall and Spring semesters.
	It shall also meet on the call of the President or by petition of one-quarter of the voting
	members of the General Assembly.

	\item The meetings of the General Assembly shall be limited to 120 minutes unless extended
	by a vote of two-thirds of the voting members present.

	\item \label{assembly-vp} The Vice President shall distribute the tentative agenda for the meeting no less than
	thirty-six hours prior to a scheduled General Assembly Meeting.

	\item A meeting of the General Assembly does not exist without a quorum. A quorum shall exist
	when at least half of the appointed or elected members of the General Assembly members are
	present.

	\item The meetings of the General Assembly shall be open to all members of GSEG. All attendees
	may participate in the discussions. The General Assembly may meet in a closed session
	comprising only of voting members and non-voting representatives by a vote of two-thirds of the
	voting members.

	\item Each General Assembly voting member shall have one vote in all matters. In order to pass
	the General Assembly, a motion must receive a simple majority of the votes cast unless
	specified otherwise in this Constitution. General Assembly members may appoint a proxy to participate on their behalf in a General Assembly meeting with full voting and other rights.

	\item Voting members and non-voting representatives of the General Assembly shall attend all
	General Assembly meetings. More than two unexcused absences per semester are grounds for
	dismissal and GSEG may contact the appointing organization to seek a replacement.

	\item Voting members and non-voting representatives of the General Assembly shall abide by
	this Constitution. A voting member or a non-voting representative who is found to violate
	this Constitution shall be given a notice stating the violation no less
	than five days before the following General Assembly meeting. Just cause (malfeasance,
	misfeasance, or nonfeasance) must be shown in all cases involving the dismissal of a voting
	member or non-voting representative.

	\item \label{assembly-log} The Chair for Logistics shall record the minutes of the meetings and distribute them to
	the GSEG membership.

\end{enumerate}
